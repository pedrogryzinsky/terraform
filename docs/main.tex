% !TeX root = ./main.tex
\documentclass[12pt, a4paper]{report}

%encoding
%--------------------------------------
\usepackage[T1]{fontenc}
\usepackage[utf8]{inputenc}
%--------------------------------------

%Portuguese-specific commands
%--------------------------------------
\usepackage[portuguese]{babel}
%-------

% Glossaries
\usepackage[acronym]{glossaries}
\makeglossaries{}

\newacronym{bu}{BU}{Business Unit}

%--------------------------------------
%begin document
\begin{document}

\title{Documentação Geral de Infraestrutura em Nuvem Pública da Plataforma SociaLab}
\author{SociaLab, ZRP}
\date{\today}

\maketitle{}
\tableofcontents{}

\printglossary[type=\acronymtype]

\printglossary{}


\printglossary[type=\acronymtype]

\begin{abstract}
	Este é um breve resumo do conteúdo do documento escrito em Português.
\end{abstract}

\chapter{Convenções}

Neste capítulo são abordadas as principais convenções adotadas na confecção da infraestrutura em nuvem, bem como os aspectos organizacionais envolvidos e a melhor forma de gerenciá-los.

\section{Organização}

A organização é a macroestrutura responsável pela gestão de uma ou mais contas em nuvem, podendo diferentes contas corresponderem à áreas ou \acrshort{bu}'s (\acrlong{bu}).

\section{Tags}

As tags permitem organizar recursos da infraestrutura e elementos organizacionais adjacentes.

Isso permite um controle e visão mais granulares desses recursos, como custo, além de permitir a remoção de múltiplos elementos de uma única vez, ou clarificar os responsáveis por um recurso ou organizações associadas ao recurso.

Dado isso, as tags são separadas da seguinte forma:

\begin{table}
	\begin{center}
		\begin{tabular}{ |c|c|c|c| }
			\hline
			col1 & col2 & col3 \\
			\hline
		\end{tabular}
	\end{center}
	\caption{Exemplo}
	\label{table:1}
\end{table}

A estrutura das tags segue por princípio a seguinte estrutura:

prefixo, contexto, identificador, ex. socialab:application:id
que identifica a qual aplicação a infraestrutura atual se referencia.

\chapter{Estrutura e Componentes}

\section{Visão Geral}

\section{Rede}

\section{Interface}
\section{Servidores}
\section{Serviços}
\section{Dados}
\section{Análise de Custos}

\chapter{Conclusão}

\end{document}
